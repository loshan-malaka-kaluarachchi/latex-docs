\documentclass[]{article}

% "\pagestyle{empty}" removes the page numbers
\pagestyle{empty}

% "\usepackage{}" is used to include other packages that can be called like the "\dfrac{•}{•}" function 
\usepackage{amsmath,amssymb,amsfonts}
\begin{document}
superscripts $$2x^3$$
$$ 2x^{34} $$
$$ 2x^{3x+4} $$

superscripts inside superscripts
$$ 2x^{3x^{4}+5} $$

subscripts
$$ x_{1} $$
$$ x_{12} $$

subscripts inside subscripts
$$x_{1_{2}}$$
$$x_{1_{2_{3}}}$$
$$a_0,a_1,a_2,\ldots,a_{100}$$

Greek letters
$$\pi$$
$$\Pi$$
$$\alpha$$
Area of a circle, C: $$A_{c} = \pi r^{2}$$

Trignometric functions
$$y = \sin x$$
$$y = \cos x$$
$$y=\csc \theta$$
$$y=\sin{(\cos{\theta}})$$
$$y=\sin^{-1} x$$
$$y=\arcsin x$$

Logarithm functions
$$y=\log x$$
$$y=\log_{5} x$$
$$y=\ln {x}$$

Roots
$$\sqrt{2}$$
$$\sqrt[3]{2}$$
$$\sqrt{x^{2}+y^{2}}$$
$$\sqrt{ 1+\sqrt{x}  }$$

Fractions
$$\frac{2}{3}$$
About $\displaystyle \frac{2}{3}$ of the glass is full.\\[16pt]
About $\frac{2}{3}$ of the glass is full.\\[6pt]
About $\dfrac{2}{3}$ of the glass is full.

$$\frac{\sqrt{x+1}}{\sqrt{x+2}}$$

$$ \frac{1}{1 + \frac{1}{x}}$$

\end{document}