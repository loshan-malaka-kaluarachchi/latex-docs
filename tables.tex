\documentclass[11pt]{article}
\parindent 0px
\usepackage{float}
\usepackage{amsfonts,amsmath,amssymb}
\pagestyle{empty}


\begin{document}

%$$\left( \frac{1}{1 + \left( \frac{1}{1 + x} \right) } \right)$$

Tables:\\

%Table 1
\begin{tabular}{|c||c|c|c|c|c|}
\hline
$x$ &1&2&3&4&5 \\ 
\hline
$f(x)$ &10&11&12&13&14 \\
\hline
\end{tabular}

\vspace{1 cm}
\begin{table}[H]
\centering
\def\arraystretch{1.5}
%Table 2
\begin{tabular}{|c||c|c|c|c|c|}
\hline
$x$ &1&2&3&4&5 \\
\hline
$f(x)$ &$\frac{1}{2}$&11&12&13&14 \\ 
\hline
\end{tabular}
\caption{These values represent the function $f(x)$.}
\end{table}

\begin{table}[H]
\centering
\caption{The realtionship between $f$ and $f'$.}
\def\arraystretch{1.5}
%Table 2
\begin{tabular}{|l|p{3 in}|}
\hline
$f(x)$ &$f'(x)$ \\
\hline
$x>0$ &The function $f(x)$ is increasing.The function $f(x)$ is increasing.The function $f(x)$ is increasing \\ 
\hline
\end{tabular}
\end{table}

Arrays:
\begin{align}
5x^2 - 9 &= x +3\\
5x^2 -x -12 &= 0
\end{align}

\begin{align}
5x^2 - 9 &= x +3\\
5x^2 -x -12 &= 0
\end{align}

\begin{align*}
5x^2 - 9 &= x +3\\
5x^2 -x -12 &= 0\\
&=12+x-5x^2
\end{align*}

\end{document}